\documentclass{article}
\usepackage[utf8]{inputenc}
\usepackage{graphicx}
\usepackage{subfigure}
\usepackage{color}
\usepackage{amssymb, amsmath}
\usepackage{array}
\usepackage[normalem]{ulem}
\usepackage{titling}
\usepackage{url}
\usepackage[colorlinks=true, urlcolor=blue, linkcolor=red]{hyperref}
\usepackage{float}
\usepackage{dialogue}


\usepackage{cite}
\newcommand{\subtitle}[1]{%
 \posttitle{%
 \par\end{center}
 \begin{center}\large#1\end{center}
 \vskip0.5em}%
}


\begin{document}

\title{Particle Filter Experiments}
\author{Helen Lu}
\maketitle


\section{Algorithm}
\subsection{Human-AI Mixed-Initiative Co-creation}
Mixed-initiative co-creation (MI-CC) refers to the collaborative process during which both the human and the artificial intelligence (AI) take the initiative to contribute to the creation of the end product \cite{yannakakis_mixed-initiative_nodate}. There has been extensive research on mixed initiative co-creative systems that create digital products. For example, Alveraz et. al explores MI-CC in storytelling \cite{alvarez_story_2022}. while others explore it in game design \cite{liapis_can_2016}. However, MI-CC in the realm of human-AI interaction have traditionally lacked the physical aspect. Our system is an instance of an MI-CC system that brings the creation into the physical world. Our system allows the user to interact with it through natural language dialogue and carries out actions in the physical world. By building a system with these capabilities, we hope that the user experiences our system as a creative partner instead of a tool.
\subsection{Human-Robot Creative and Physical Collaboration}
Prior research on human-robot collaboration has largely focused on physical tasks in which success is clearly defined. For examples, Admoni et al. \cite{admoni_robot_2016} and Shah et al. \cite{shah_improved_2011} focus on assembly tasks, Vasey et al. \cite{vasey_collaborative_2016} focus on a construction tasks, and Hinds et al. \cite{hinds_whose_2004} focus on sorting tasks. However, We are interested in creative tasks in which the the goals of the tasks are loosely defined thereby giving the human more freedom to complete the tasks according to their preferences. More specifically, we want to focus on human-robot artistic co-creation tasks during which both the human and the robot participate in the ideation process.

% \begin{figure}[ht]
%     \centering
%     \includegraphics[width=\linewidth]{images/Cake_and_decorations.png}
%     \caption{The cake and decorations}
%     \label{fig:exp_setup}
% \end{figure}
\section{Results}



\end{document}
